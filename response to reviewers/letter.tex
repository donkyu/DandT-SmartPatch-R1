%\documentclass{article}
\documentclass[onecolumn]{IEEEconf}
%\usepackage{a4wide}
\usepackage{enumitem}
\usepackage[usenames, dvipsnames]{color}
\usepackage{graphicx}
\usepackage{subcaption}
\usepackage{ulem}
\renewcommand{\figurename}{Fig.}
\renewcommand*{\thetable}{\Roman{table}}

\title{Response to the Reviewers' Comments}
\begin{document}

\maketitle

The authors would like to thank the reviewers for the  comments and suggestions on the technical contents and presentation of our manuscript ``SmartPatch: A Self-Powered and Patchable Cumulative UV Irradiance Meter,'' submitted to IEEE Design and Test. This revised manuscript has been greatly improved thanks to the reviewers' invaluable advices. We have revised the manuscript faithfully following the reviewers' comments. We include the newly added or significantly modified parts in the revised version of the paper also in Response to Reviewers. We highlight the important technical content changes and set those parts in a red color in the revised paper. Detailed comments and corresponding corrections are listed below:\\

\setlength{\parindent}{0cm}
%%%%%%%%%%%%%%%%%%%%%%%%%%%%
\textbf{General Revision:}
We found that several important comments raised by the reviewers were due to misunderstanding of the paper contents. Most importantly, it is because our presentation was not proper enough. We completely rewrite Abstract and Introduction to deliver the technical contents more clearly. \\

\textbf{Reviewer 1:}
%%%%%%%%%%%%%%%%%%%%%%%%%%%%
\begin{description}
\item [C1: ] Is the number of modes a significant limitation of SmartPatch? It would also help to understand this if the amount MED varies with skin type is shown. If MED varies significantly between pale and very dark skin, and since SPF is multiplicative with maximum UV exposure time I would assume that the four modes does not cover the user space well. You could easily have more than two bins for skin type and way more sunscreen options (SPF 30+) that would significantly alter maximum UV exposure time.  
\item [R1: ] We will revise the paper contents as follows:
\begin{itemize}
\item We modify the terms as follows: operating mode and parameter setting.  The "mode" used in the previous manuscript is confusing, and thus we change it to parameter setting. The method for parameter setting in the previous manuscript has "state explosion" issue for the user interface because the number of parameters (mode in the previous manuscript) increase by the number of skin types multiplied by the number of SFP types. So, we revised the design to accommodate easy handling of up to 10 skin types or more, and up to 10 SFP types or more.
\item The operating modes are reset, measure, skin type setting, and SPF setting.
\item We revise Fig.~5 accordingly.
\end{itemize}
\item [C2: ] ``The existing UV meters under- or overestimates the accumulation of UV irradiance.'' I would assume that existing UV meters are accurate with respect to the part of the body it is monitoring. This should be corrected. In addition, is the shoulder the most appropriate place to measure the UV for these activities (e.g., chatting on a bench, walking outdoors, seated on a grass field)? For a person that is wearing a typical shirt, is the UV measurements in Fig. 9 given for their shoulder under their shirt? Or is the use-case more specific than that (e.g., tank tops or shirt-less)?
\item [R2: ] 
Existing UV irradiance meters can be accurate if their UV sensors are positioned on the exact location of a particular skin with the same perpendicular angle to the Sun. However, due to the inherently limited form factor, weight and volume of existing UV meters, most of them only measure the UV irradiance on limited positions of the human body (e.g., wrist and chest.) In addition, they have different perpendicular angles to the Sun, and the angles are changed significantly depending on the moving and activity. Therefore, the measurement can be under or overestimated by the skin area. For example, an wrist strap type UV meter underestimates the accumulation of the UV irradiation on the shoulder.\\ 
~\\

\item [C3: ] NVRAM should be defined at its first use.
\item [R3: ] We defined NVRAM (the non-volatile RAM) in the third row of the third paragraph in Section III-A as following:

\textit{The power management unit (PMU) detects power interruption and allows the non-volatile RAM (NVRAM) to store the necessary context [10].}\\
~\\

\item [C4:] The English and phrasing should be carefully reviewed. For example, it was difficult to parse ``We measure the UV irradiance on the wrists, chests… as a reference with several UV irradiance meters including SmartPatch (Fig. 2.).'' 

\item [R4: ] We revised the whole manuscript, and we colored modified sentences. \\

\underline{We modify Section III-C in the manuscript as:}\\
\textcolor{red}{To verify the functionalities and usefulness of SmartPatch, we compare the UV irradiance on various skin areas in the morning and afternoon as shown in Fig.~7. We measure the UV irradiation on the wrists, chests, shoulders, and from the ground as a reference.} The skin area of interest in the experiment is the shoulder, and SmartPatch is attached to the shoulder skin. Other UV meters are carried or attached as designed. For instance, a watch type UV meter measures the UV irradiance on the wrist. The UV irradiance to the ground increases by the angle to the Sun that is the maximum at 1:36 pm (13:36) in the experiments.\\
~\\


\item [C5:] Also, ``In cooperation with a sunscreen with a specific SPF and skin type, which are user programmable'' seems to imply that users can program their sunscreen's SPF and change their skin type and cooperation would imply that sunscreen can make their own actions. I believe it should be more along the lines of: SmartPatch can aid the user in when to reapply sunscreen based on the user's skin type and sunscreen SPF.

\item [R5: ] \underline{Your response is totally nonsense.} The answer should be "We are sorry for confusing sentence, which reads funny. We apologize such mistake. We rephrase the sentence as follows:"\\ \\
1. SmartPatch let users know the remaining exposure time not the time to reapply the sunscreen. The sunscreen is recommended to be reapplied every two hours because the sunscreen is easily washed away by water or sweat. The sunscreen reduces the chance of skin damage when it is perfectly applied to the skin. Please see the following description in the fifth paragraph in Section II-A:\\

\textit{A bigger number of SPF \textcolor{red}{implies a higher reduction rate} of UV absorption and \textcolor{red}{efficiently} reduces chances of skin damage. For example, a sunscreen with SPF 15 blocks 93\% of UV irradiance and extends the time to produce erythema about 15 times longer. \textcolor{red}{Such a high rate of UV filtering certainly extends UV exposure time without skin damage, UV irradiance is still accumulated in the skin even with a sunscreen regardless of the SPF, and it may eventually incur skin damage.}}\\

2. The remaining exposure time not to receive skin damage is calculated based on the sunscreen SPF and user's skin type, and these parameters are programmable in SmartPatch. We modify the beginning of Section III in the manuscript as:

\begin{itemize}
\item Having a small form factor and possibly disposable,
\item \textcolor{red}{Programming the sunscreen SPF and user's skin type,}
\item Calculating the maximum UV exposure time based on the UV irradiance and skin type,
\item A patchable design to measure the actual cumulative UV irradiance on the exact skin area of interest,
\item Self-powered by a PV cell with the minimum possible size, and
\item No use of a power converter nor significant energy storage.
\end{itemize}

\end{description}

\pagebreak


%%%%%%%%%%%%%%%%%%%%%%%%%%%%
\textbf{Reviewer 2:}
%%%%%%%%%%%%%%%%%%%%%%%%%%%%
\begin{description}
\item [C1: ] In abstract, you should address the motivation, your approach, and experimental results.

\item [R1: ] IEEE Design and Test articles do not have Abstract unlike Transactions papers. The Associate Editor adds his/her Editor's note, which would be short Abstract. Nevertheless, as the reviewer suggested, we added Abstract as follows, which can be a source of Editor's note:

\textcolor{red}{Ultraviolet (UV) irradiance affects human bodies both positively and negatively. We introduce SmartPatch, a self-powered, small-form-factor, light-weight, low-cost, and a patch-type UV meter, that provides a scientific measure of UV radiation and irradiation on a particular skin area. It is powered by a tiny PV (photovoltaic) cell without a battery and a power converter and performs UI (user interface) without a physical switch.}\\
~\\

\item [C2: ] In Introduction, you can state your approach and the comparisons between your approach and the existing state-of-the-art approach.
\item [R2: ] As the reviewer suggested, we totally revise Introduction (Section I) to state our approach: 1) what normal people usually do to avoid excessive UV irradiance, 2) why we need more scientific information to consider individual's characteristics and 3) brief introduction to SmartPatch. In addition, we revise the backgrounds (Section II-B) to compare our implementation, SmartPatch, with the existing state-of-the-arts.\\

\uline{We modify the second paragraph in Section I to state 1) what normal people usually do to avoid excessive UV irradiance:}\\
%
\textcolor{red}{The intensity of UV irradiance is often quantified by the UV index (UVI). Region-based daily UVIs are commonly broadcasted through weather forecast channels. There are general guidelines to avoid skin damage classified with the UVI as shown in Fig.~1. For example, we are recommended to stay indoor when the UVI  is over 8.}\\

\uline{We modify the third and fourth paragraphs in Section I to state 2) why we need more scientific information considering individual's characteristics:}\\
%
\textcolor{red}{In order to safely perform outdoor activities without experiencing skin damage, we should be aware of  the maximum UV irradiation. UVI and UV exposure time are easy metrics to estimate UV irradiation for normal people when a medical-grade scientific measure is not accessible. Although we commonly estimate the UV irradiation by the UV exposure time, it cannot be a reasonable estimation of UV irradiation on a particular skin area because UV irradiation is determined by the time integral of the instantaneous UVI accounting the angle between the Sun, in addition to the individual factors such as skin color. Additional UV protection such as sunscreen lotion makes it even more difficult.} 

\textcolor{red}{It is crucial to directly measure the UV irradiance and irradiation on the target skin surface of interest, \textit{i.e.} a UV sensor must be directly mounted on the skin surface of interest with the same angle to the Sun. The UV meter continuously reads the UV sensor and integrates the value over time. Unfortunately, to the best of our knowledge, existing UV measuring tools for normal people (excluding a laboratory measurement setup) can hardly achieve this goal.}\\

\uline{We modify the fifth and sixth paragraphs in Section I to state 3) introduction to SmartPatch a solar powered patchable smart UV level meter:}\\
%
\textcolor{red}{In this article, we introduce SmartPatch, a solar-powered and patchable smart UV irradiation meter that informs both the current UV irradiance and irradiation. SmartPatch is designed to attach on the skin surface of interest, its UV measurement meets the above mentioned requirements such as the angle to the Sun, environmental change and body movement. Two key technologies are behind the proposed SmartPatch. The first is a storage-less and converter-less energy harvesting~[9] that does not require a battery nor a voltage converter for solar energy harvesting. The second is a switch-less user interface. These two technologies can significantly reduce both volumetric and gravimetric overheads as well as the manufacturing cost.}

\textcolor{red}{SmartPatch detects the correlation between the UV irradiance and solar irradiance. Natural operation keeps a high correlation between them. The correlation is broken if the UV sensor is intentionally blocked by a finger. SmartPatch detects such finger actions and utilizes them for UI (user interface) as if physical switches are pressed by the finger. A smartphone LED flashlight can do the same job. We design an app that  generates pre-defined light flashing patterns so that a single screen touch can replace multiple times of finger actions. These technologies make it possible to implement SmartPatch with simple integrated circuits in a single chip, and also make it possible for a tiny PV cell to directly supply power to the chip without a power converter and a battery. As a result, SmartPatch becomes extremely low-cost, low profile, small, and light.}

 \textcolor{red}{The main functions of SmartPatch include 1) displaying the current UVI, 2) displaying the remaining UV irradiation to avoid skin damage, and 3) mode change or parameter setting for the personalized skin type and the SPF (Sun protection factor) of the currently applied sunscreen lotion. We verified the functionalities and usefulness of SmartPatch performing various outdoor activities.}\\

\underline{We modify Section II-B to compare our implementation, SmartPatch, with the existing state-of-the-art:}\\
\textcolor{red}{There are various portable UV meters on the market as shown in Fig.~2(a). Most UV meters can only measure the instantaneous UVI though we emphasize UV irradiation measurement is meaningful. Some top-of-the-line UV meters (Fig.~2(b)) additionally inform UV irradiation. They are capable of notifying the maximum allowable UV exposure time based on the instantaneous UVI, skin type and SPF. In other words, the feature basically meets the requirement.  First, however, these devices typically include a battery~[5], [6]. The types of such class devices are often a wrist strap, a watch, a pendant, or a badge. So, second, such devices can only measure the UV irradiance of limited positions of the human body, which is unlikely vulnerable to sunburn.} 
%
As shown in Fig.~3, each skin surface area has a distinctly different amount of UV irradiance due to the different angle to the Sun. In addition, partial shading can continuously occur on each skin surface area by trees, buildings, other body parts, clothes, and so forth. For instance, people often experience skin damages on their shoulders while other skin areas (\textit{e.g.}, a wrist that has the UV strap) are manageable even if the whole body has been exposed to the Sun.

\textcolor{red}{Therefore, an accurate UV irradiation meter should position the UV sensor on the exact location of the target skin surface with the same perpendicular angle to the Sun. However, it is not practical to mount a separate UV sensor from the \textcolor{red}{UV meter main unit in Figs.~2(a) and~2(b)}. Therefore, a patchable design of a UV irradiance meter is only capable of measuring correct UVI. We easily figure out the most vulnerable skin area and attach the device onto it taking into account the environment, our outfit and activities. Recently, a patchable UV meter has been introduced by a cosmetic company as shown in Fig.~2(c)~[7]. It is powered by the user's smartphone via Near Field Communication (NFC.) However, the device should equipped with a large-size energy storage such as a supercapacitor and thus a power converter to continuously measure the UV irradiation after it has received energy from NFC.}\\
~\\

\item [C3: ] In Section II, authors state that "However they (devices) does not show the exact amount of accumulated UV irradiance." \\
1. Maybe the user does not care the exact amount of UV. \\
2. How do you know these devices cannot measure the exact amount? Do you have any literatures to support it?\\
3. In experiments, you also can compare your device with the existing devices to show how accuracy your device is.
\item [R3: ] 
1. We are sorry for the confused sentence. For clear understanding, we have significantly revise this section.\\
2. Typical UV meters cannot be attached on the skin area of interest, and they cannot measure the accurate amount of UV irradiance on that area.\\
3. To verify the functionalities and usefulness of SmartPatch, we compare the UV irradiance on various skin areas in different times of a day (in the morning and afternoon) as shown in Fig.~7. We measure the UV irradiation on the wrists, chests, shoulders, and from the ground as a reference. Please see fourth, fifth and sixth paragraphs in Section IV:\\

\textit{\textcolor{red}{To verify the functionalities and usefulness of SmartPatch, we compare the UV irradiance on various skin areas in the morning and afternoon as shown in Fig.~7. We measure the UV irradiation on the wrists, chests, shoulders, and from the ground as a reference.}
The skin area of interest in the experiment is the shoulder, and SmartPatch is attached to the shoulder skin. Other UV meters are carried or attached as designed. For instance, a watch type UV meter measures the UV irradiance on the wrist. The UV irradiance to the ground increases by the angle to the Sun that is the maximum at 1:36 pm (13:36) in the experiments.\\
%
Of course, the UV irradiance to the skin area is different by activities.
For example, playing soccer, working and walking outdoor cause high UV irradiance to the shoulder and chest as shown in Fig.7(a).
Unfortunately, the watch type UV meter on the wrist does not reflect the UV irradiance variation on the skin area of interest, i.e., shoulders, by the activities and the angle to the Sun.
However, it is impossible to put the watch type UV meter on the shoulder.
The UV irradiance to the shoulder is sometimes even larger than the UV irradiance to the ground when the shoulder has a perpendicular angle to the Sun.\\
%
We observe that existing UV meters under- or overestimate the accumulation of UV irradiance.
This implies that watch type UV meters underestimate UV accumulation on the shoulder up to 16\%, which may cause 50\% more chances of erythema symptoms if UV exposure lasts until the watch type UV meters indicate the maximum exposure time is over~[3].}\\
~\\

\item [C4: ] Also in Section II, the authors state that the reason why these devices are inaccurate is because every skin surface has a different perpendicular angle to the Sun. In your device, the reviewer wonders how you address this issue.
\item [R4: ] We implement a patchable UV level meter to directly attach on the skin area of interest. Two key technologies are behind the proposed SmartPatch to be a patchable device. The first is a storage-less and converter-less energy harvesting~[1] that does not require a battery nor a voltage converter for solar energy harvesting. The second is a switch-less user interface. These two technologies can significantly reduce both volumetric and gravimetric overheads as well as the manufacturing cost. To make the uniqueness of our work clear, we revised Introduction as follow:\\

\underline{We modify the fifth, sixth and seventh paragraphs in Section I in the manuscript as:}\\
\textcolor{red}{In this article, we introduce SmartPatch, a solar-powered and patchable smart UV irradiation meter that informs both the current UV irradiance and irradiation. SmartPatch is designed to attach on the skin surface of interest, its UV measurement meets the above mentioned requirements such as the angle to the Sun, environmental change and body movement. Two key technologies are behind the proposed SmartPatch. The first is a storage-less and converter-less energy harvesting~[9] that does not require a battery nor a voltage converter for solar energy harvesting. The second is a switch-less user interface. These two technologies can significantly reduce both volumetric and gravimetric overheads as well as the manufacturing cost.}

\textcolor{red}{SmartPatch detects the correlation between the UV irradiance and solar irradiance. Natural operation keeps a high correlation between them. The correlation is broken if the UV sensor is intentionally blocked by a finger. SmartPatch detects such finger actions and utilizes them for UI (user interface) as if physical switches are pressed by the finger. A smartphone LED flashlight can do the same job. We design an app that  generates pre-defined light flashing patterns so that a single screen touch can replace multiple times of finger actions. These technologies make it possible to implement SmartPatch with simple integrated circuits in a single chip, and also make it possible for a tiny PV cell to directly supply power to the chip without a power converter and a battery. As a result, SmartPatch becomes extremely low-cost, low profile, small, and light.}

 \textcolor{red}{The main functions of SmartPatch include 1) displaying the current UVI, 2) displaying the remaining UV irradiation to avoid skin damage, and 3) mode change or parameter setting for the personalized skin type and the SPF (Sun protection factor) of the currently applied sunscreen lotion. We verified the functionalities and usefulness of SmartPatch performing various outdoor activities.}\\
~\\

\item [C5: ] In experimental results, you should compare your results with the existing approaches and see what your improvements are in terms of power reduction and accuracy.
\item [R5: ] 
1. As mentioned in R3-3, we describe how our device is accurate compared with other existing approaches in Section IV in the manuscript:\\

\textit{We observe that existing UV meters under- or overestimate the accumulation of UV irradiance.
This implies that watch type UV meters underestimate UV accumulation on the shoulder up to 16\%, which may cause 50\% more chances of erythema symptoms if UV exposure lasts until the watch type UV meters indicate the maximum exposure time is over~[3].}\\

2. The power consumption of our product is low enough because the measured power consumption is lower than harvested power with the storage-less and converter-less energy harvesting technique. Please see the third paragraph in Section IV:\\

\textit{Finally, we measure the power consumption of the prototype including the ASIC. Table~III summarizes the power consumption of each component. The ASIC itself consumes 1.6 mW while the other peripherals consume 6.4 mW. In total, the prototype consumes 8 mW, which is low enough to use a small size PV cell (22 mm by 7 mm, 12.92 mW@$V_{MPP}$-3.4 V.) The final implementation will have a single chip ASIC including the NVRAM, an e-ink display and the optimal-size of PV cell on a flexible PCB. This is being lead by a company through technology transfer.}

\end{description}

\end{document}



